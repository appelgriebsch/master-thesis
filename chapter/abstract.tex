%!TEX root = ../MasterThesis.tex

\chapter*{Abstract}

There is a dramatic shift in credit card fraud from the offline to the online world. Large online retailers have tried to establish countermeasures and transaction data analysis technologies to lower the rate of fraudulent transactions to a manageable amount. But as retailers will always have to make a trade-off between the \textit{performance} of the transaction processing, the \textit{usability} of the web shop and the overall \textit{security} of it, one can assume that \gls{E-commerce} fraud will still happen in the future and that retailers have to collaborate with relevant business partners on the incident to find a common ground and take coordinated (legal) actions against it. \\

Trying to combine the information from different stakeholders will face issues due to different wordings and data formats, competing incentives of the stakeholders to participate on information sharing as well as possible sharing restrictions, that prevent them from making the information available to a larger audience. Additionally, as some of the information might be confidential or business-critical to at least one of the parties involved, a \textit{centralized} system (e.g.\ a service in the cloud) can \textbf{\underline{not}} be used. \\

This Master thesis is therefore analysing how far a computer supported collaborative work system based on peer-to-peer communication and Semantic Web technologies can improve the efficiency and effectivity of \gls{E-commerce} fraud investigations within an inter-institutional team. \\[2em]

\textbf{Keywords:} Peer-To-Peer Communication, Semantic Web, CSCW
