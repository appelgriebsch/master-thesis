%!TEX root = ../MasterThesis.tex

\chapter*{Abstract}

There is a dramatic shift in credit card fraud from the offline to the online world.
Large online retailers have tried to establish countermeasures and transaction data analysis technologies
to lower the rate of fraudulent transactions to a manageable amount. But as retailers will always have to
make a trade-off between the \textit{performance} of the transaction processing, the \textit{usability} of the web shop
and the overall \textit{security} of it, we can assume that E-commerce fraud will still happen in the future and that
retailers have to collaborate with relative parties on the incident to find a common ground on and take coordinated
(legal) actions against it.
\\[0.8em]
Combining information from different stakeholders will face issues due to different wordings and data formats of the
information, competing incentives of the stakeholders to participate on information sharing as well as possible sharing
restrictions, that prevent making the information available to a larger audience. Additionally, as some of the information
might be confidential or business-critical to one of the involved parties a \textit{centralized} system
(e.g.\ a service in the cloud) could \textbf{\underline{not}} be used.
\\[0.8em]
This Master thesis is therefore looking into the topic of how far a computer supported collaborative work system
based on peer-to-peer communication technologies and shared ontologies can improve the efficiency and effectivity
of E-commerce fraud investigations within an inter-institutional team.
\\[2em]
\textbf{Keywords:} Peer-To-Peer Communication, Semantic Web, CSCW
