%!TEX root = ../MasterThesis.tex

\section{\gls{E-commerce} fraud incidents}
\label{sec:scenario_fraud}

Based on the previous sections one can come up with strategies fraudsters might use to trick the \gls{E-commerce} system. To do so the criminals will have to get access to credit card information in the first place. Therefore this section first looks into ways a criminal might get access to credit card and personal related information in the \gls{E-commerce} scenario. After that the section describes possible strategies fraudsters can use to cheat the system. The section ends with a discussion of the \gls{E-commerce} fraud incident handling as it is in place today.

\subsection{Credit Card data breaches}
\label{subsec:leaking_credit_cards}

Based on information in the Section~\ref{sec:stakeholder_data_flow} one can figure out the parties, who have access to or store credit card information in the \gls{E-commerce} scenario, namely:\@

\begin{itemize}
  \item a consumer as owner of a credit card,
  \item an issuer, who handed out a credit card to a consumer,
  \item a merchant, if a consumer is paying with credit card,
  \item a Payment Service Provider, if a consumer is paying with a credit card online.
\end{itemize}

The \gls{PSP}s receive credit card information from merchants with the payment authorization requests. If the \gls{PSP}s do the authorization themselves, they are also the participants, who store and hold the credit card information in their back-end databases. As mentioned earlier the \gls{PSP}s should follow industry standards and guidelines for storing and processing payment-related information; especially the \gls{PCI/DSS} standard \citep{virtue2009payment}. In addition they are responsible for monitoring their systems with an intrusion detection program. This utility will trigger a signal as soon as an hacker got access to the internal databases. In that case the \gls{PSP}s can put the leaked credit card information on an internal blacklist, so that these cards can no longer be used for further payments online. Additionally they will have to send a message to the corresponding issuers, to which the \gls{PSP}s generally maintain strong business relationships. The issuers will inform the affected credit card owners and send out a new credit card to each of them. Due to this procedure in place, one can assume that the safety and security of credit card handling at the \gls{PSP}s can be guaranteed. \\

The merchants receive the credit card information during the checkout processes from the consumers. The credit card information are transfered via the public Internet from the consumers to the merchants and could be victims to man-in-the-middle attacks, in which hackers are intercepting the communication between the consumers and the merchants with the objectives to capture the personal and payment-related information from the data transmission streams. Therefore the merchants should offer their Web shops via a secure communication channel only. For that they can use industry standards such as \gls{TLS} to encrypt the information that is send between both parties. Doing so will make it more difficult for attackers to get to the plaintext information exchanged between consumers and merchants during the checkouts. As the merchants are not processing the credit card information directly, they also do not have to store them in their own back-end databases. The merchants are asking the \gls{PSP}s or the issuers of the credit cards for authorization of a payment and receive an unique payment token in response, if that authorization was successful. As stated in the \gls{PCI/DSS} standard \citep{virtue2009payment} merchants should \emph{never} store credit card information as a whole in their own databases, but should use the unique payment tokens and shortened credit card data (especially abbreviated credit card numbers) to refer to a specific payment later. Due to this procedure in place one can conclude that breaking into the systems of a merchant will not result in any leaked credit card information, if the merchants follow these guidelines. \\

The issuers are a valuable target for hacking into the back-end systems with the objective to leak a massive amount of credit card and personal related information. As financial institutions the issuers also have to follow a huge set of regulations and safety procedures to be able to participate on the market. It can be assumed that at least the same safety mechanisms are valid as are in place for the \gls{PSP}s. This means constantly monitoring the internal systems with an intrusion detection mechanism and blacklisting any leaked credit card. In addition to the monitoring of all online activities (as also the \gls{PSP}s are doing) the issuers can monitor activities done with the credit card in the offline world too. In case of suspicious activities the credit cards can be blocked immediately, and new ones will be send out to each affected owner. \\

The consumers are also a valuable target for eavesdropping on credit card and personal related information. They are also the weakest and most insecure party in the whole E-commerce scenario. As shown before a lot of the protection mechanisms of the other participants rely on following industry standards, and on constantly monitoring the own systems for malicious activities. This can not be securely said about the computers of the consumers though. Whether they are using up-to-date security programs (e.g.\ an Anti-virus tool and a firewall) on their computers or not is out of reach of the other actors to verify. Additionally, consumers can fall victim to phishing attacks, which will send them to malicious Web sites with the intend to get their personal related information. In some seldom cases the consumers might cooperate with fraudsters, or might be the impostors themselves with the intent to trick the system for their self-interests. Due to these facts an \gls{E-commerce} fraud investigation can not rely on information from the consumers at all, but instead has to figure out if a suspicious transaction was triggered from the owner of the credit card, or if the transaction was coming from a deceiver.

% subsection leaking_credit_cards (end)

\subsection{E-commerce fraud strategies}
\label{subsec:strategies_fraudster}

After fraudsters have got access to leaked credit card information they can come up with the following strategies to trick the \gls{E-commerce} system:\@

\begin{enumerate}
  \item a deceiver owns information about \textbf{one} leaked credit card and try to use it for ordering products from \textbf{multiple} merchants on the Internet,
  \item a deceiver owns information about \textbf{multiple} leaked credit cards and try to use them for ordering products from \textbf{one} merchant on the Internet,
  \item a combination of the two cases above, which can also be related to as a series of the first fraud activity.
\end{enumerate}

In the first scenario, in which the fraudsters try out a leaked credit card for ordering products on Web shops of various merchants, each of the merchants only sees the transaction that takes place in their systems. This will make it more difficult for merchants to detect whether there are fraudulent transactions or not, because they are not aware of the attempts the fraudsters did on other merchant's Web shops. \\

As each merchant will rely on a \gls{PSP} or an issuer to verify the credit card payment, it is in the responsibility of these participants to recognize fraudulent transactions in this specific scenario. To be able to do so, the \gls{PSP}s and also the issuers are monitoring the usage of credit cards and are actively looking for suspicious activities. The fraud prevention mechanisms in place are mostly working on rule-based, and in some cases also on score-based systems running in the internal networks of the \gls{PSP}s and the issuers. These systems are fed with the information the merchants send with the payment authorization requests and will come up with a decision on each transaction that is either:\@

\begin{enumerate}
  \item Yes, this looks like a fraudulent transaction and has to be blocked.
  \item No, this seems to be a valid transaction and should be acknowledged.
  \item Maybe, this transaction might be valid, but there is some uncertainty in the validation of it. These edge cases are routed to a human operator of a \gls{PSP} or an issuer to decide on how to proceed with them.
\end{enumerate}

As a recent study shows the success rate of the fraud prevention systems heavily relies on the techniques used to validate the transaction data \citep{rana2015survey}. The outcome is that ca. 70 to 80\% of the fraudulent transactions will be currently recognized as such and blocked successfully. That still means up to 30 percent of fraudulent transactions could not be identified correctly. For handling these edge cases each organization employs special trained staff, which is operating 24/7 and 365 days a year, for handling them. \\

As stated in the introductory of this Master thesis in Chapter~\ref{cha:introduction}, there is a shift from the offline credit card fraud to the online world. This is also resembled in current figures of \gls{E-commerce} fraud incidents, which show that it makes up to 85 percent of all credit card fraud attempts and have on average a transaction value of 500 to 600 EUR.\\

As the \gls{PSP}s and the issuers do not have any order details, they can only decide on the information given during the payment authorization requests (see Section~\ref{sec:stakeholder_analysis}). At most they can validate the branch a merchant is operating in, and it might come as no surprise that the fraudsters are regularly using Web shops of merchants, who offer either electronics, clothings, entertainment- or travel-related products and services. These are also the most commonly used sources of \emph{valid} \gls{E-commerce} transactions, and will therefore make any fraudulent transaction very difficult to detect. \\

At the end it might be the owners of the credit cards, who detect suspicious activities on their credit card accounts and inform their issuers about them. Based on current regulations and laws the issuers have to rollback the fraudulent transactions on request of the consumers, which means that the merchants will have to cover the costs of the \gls{E-commerce} frauds (as they are not receiving the money for the products that might have been shipped to the fraudsters already). \\

Looking at the second scenario of the \gls{E-commerce} fraud strategies at the beginning of this section, a merchant will receive multiple requests from a deceiver, who is trying out various leaked credit cards for finishing an order. These kind of \gls{E-commerce} frauds can be recognized at the systems of the merchants based on the same source \gls{IP} address of the requests, or due to having the same shipping address for orders with different credit cards. Therefore, one can conclude that also merchants must take an active role in the fraud prevention processes (if they do not do so already) and try to minimize the amount of fraudulent transactions taken place in their Web shops. As this scenario is likely be manageable with additional fraud prevention mechanisms at the merchants, and does not need to involve other parties of the \gls{E-commerce} scenario to figure out the validity of the transactions, this second scenario falls out of scope of this Master thesis.

% subsection strategies_fraudster (end)

\subsection{Handling of E-commerce fraud incidents}
\label{subsec:e_commerce_fraud_handling}

If the fraud prevention systems at the \gls{PSP}s or the issuers are detecting a suspicious transaction, an operator working in a special department within the organization will be informed about that transaction via a notification on his or her computer. This operator will have to decide whether the transaction looks valid and should be acknowledged, or seems to be fraudulent and has to be denied. To be able to decide this, he or she is going to look into the recent usages of the credit card in question. Whereas it will be easy to recognize that a credit card, which was just being used in a shop in Germany, could not be used in a shop in US or Asia within a short time-frame due to physical constraints in the real world, the same consumer can order products from an US or Asian online retailer with ease within minutes. So these initial geographical constraints, which work so well with real-world usage patterns of credit cards (a proven fraud prevention mechanism called Geo-fencing), will no longer work in the \gls{E-commerce} scenario. Also blocking incoming authorization request based on the geographical location, which is related to the \gls{IP} address of the consumer, will not be sufficient as deceivers are able to fake them with ease. \\

So the operators have to found their decisions on the transaction information at hand. Initially they can check for the amount that has been paid with the credit card. One can assume that small amounts will be covered by the \gls{PSP}s or the issuers, who will take over the risk for a false payment authorization. But with an increased value of the items ordered, the \gls{PSP}s and the issuers are putting back the risk to the merchants in case of any customer complaints later. At a second glance the operators can also verify whether a customer has had any business relationship with a merchant in the past or not, as well as check for the retail branch a merchant operates in. But these are weak hints for investigating the validity of an \gls{E-commerce} transaction as they can be bypassed by the fraudsters with ease (see the explanations in the previous section). \\

To make a solid decision the operators will have to get in contact with all the merchants a credit card has been used with recently, and have to ask for additional information such as:\@

\begin{itemize}
  \item Does the consumer owns an user account with the merchant's Web shop?
  \item What is the consumer usually looking for in the merchant's Web shop?
  \item Does the shipping address matches the billing address for that order?
  \item If not, has the consumer send orders to this shipping address in the past?
  \item What has been ordered by the consumer, \gls{incl}\ detailed product information such as brand, model, product categories, \ldots?
\end{itemize}

In some cases the \gls{PSP}s or the issuers have had a business relationship with an online merchant in the past. So the operators from the \gls{PSP}s or the issuers might already know whom to contact from the support personnel of that merchant. But in most cases the contact persons might not be known to them, so they have to send a request to the general support staff via the contact forms on the merchant's Web site. \\

Getting the right information will still take time, because the correct addressees from the support departments of the merchants are unknown, the merchants do not have specialized staff at hand to handle these kind of inquiries, or there might be misunderstandings on handling a request due to language barriers or different incentives between the participants. Additionally the operator, who is responsible for such a case, has to collect all available information from these merchants, notes them down and tries to build a ``big picture'' out of them. In case the initial information received from one of the participants have not been enough, the operator will have to get in contact with the support personnel again. This can result in a lengthy sequence of communication attempts and question-response processes between an operator and the online merchants concerned. Due to this, getting an in-depth overview of suspicious credit card usages in the \gls{E-commerce} scenario is likely taking hours if not days or weeks. That is definitely way to much time and effort to look into any of these suspicious transactions in detail. Therefore one can assume that an in-depth analysis of any suspicious transaction will not take place today; instead most of these transactions will be acknowledged without any doubt after a first short look and plausibility check by the operators. \\

Still the merchants as well as the \gls{PSP}s and the issuers have a high incentive for increasing the success rates of their fraud prevention mechanisms, and for keeping the numbers of successful fraudulent activities low. For the \gls{PSP}s and the issuers there are regulations stating that at maximum only one thousands of the overall transactions\footnote{Note: numbers stated are valid for the EU.} can be fraudulent. This keeps the pressure on these financial institutions to invest in fraud prevention techniques for being able to stay in business. For the merchants it is also of high interest, that a fraudulent transaction can be resolved before a deceiver receives the ordered products. In the worst case scenario just \emph{one} successfully performed fraudulent transaction in an \gls{E-commerce} shop will trigger hundreds if not thousands of subsequent attempts from other fraudsters, as past experiences have shown.

% subsection e_commerce_fraud_handling

% section scenario_fraud (end)
