%!TEX root = ../MasterThesis.tex

\section{E-commerce fraud incidents}
\label{sec:scenario_fraud}

Based on the previous sections one can come up with strategies a fraudster might use to trick the E-commerce system. To do so the criminal will have to get access to credit card information in the first place. Therefore this section first looks into ways a criminal might get access to credit card and personal information in the E-commerce scenario. After that the section describes possible strategies a fraudster can use to trick the system. The section ends with a discussion of the E-commerce fraud incident handling as it is in place today.

\subsection{Credit Card data breaches}
\label{subsec:leaking_credit_cards}

 In the Section~\ref{sec:stakeholder_data_flow} one could already figure out the parties, who have access to or store credit card information in the E-commerce scenario, namely:\@

\begin{itemize}
  \item the consumer as owner of the credit card
  \item the issuing bank, who handed out the credit card to the consumer
  \item the merchant, if the consumer is paying with a credit card
  \item the Payment Service Provider, if the consumer is paying with a credit card online
\end{itemize}

The \textbf{\gls{PSP}} does receive the credit card information from the merchant during the authorization of the payment. If the \gls{PSP} does the authorization herself, she is also the participant, who stores and holds the credit card information in her backend databases. As mentioned earlier the \gls{PSP} should follow industry standards and guidelines for storing and processing payment-related information; especially the PCI/DSS standard \citep{virtue2009payment}. In addition she is responsible for monitoring her systems with an intrusion detection program. This will trigger a signal as soon as an hacker got access to the internal databases. In that case the \gls{PSP} can put the leaked credit card information on an internal blacklist, so that these cards could no longer be used for further payments online. Additionally she will have to send a message to the corresponding issuers, to which the \gls{PSP} generally maintains a strong business relationship. The issuer will inform the affected credit card owners and send out a credit card to each of them. Due to this procedure in place, one can assume that the safety and security of credit card handling at the \gls{PSP} can be guaranteed. \\

The \textbf{merchant} receives the credit card information during the checkout process from the consumer. The credit card information is transfered via the public Internet from the consumer to the merchant and could be a victim of a man-in-the-middle attack, in which the hacker is intercepting the communication between the consumer and the merchant with the objective to capture the personal and payment-related information from the data transmission stream. Therefore the merchant should offer the Web shop via a secure communication channel only. For that she can use industry standards such as TLS to encrypt the information send between both parties. This will make it more difficult for an attacker to get to the plaintext information exchanged between consumer and merchant during the checkout process. \\

As the merchant is not processing the credit card information directly, she also do not have to store them in her own backend databases. The merchant is asking the \gls{PSP} or issuer of the card for authorization of the credit card payment and receives an \textbf{unique payment token} in response, if the authorization was successful. As stated in the PCI/DSS standard \citep{virtue2009payment} a merchant should never store the whole credit card information in her databases, but should use this unique payment token and shortened credit card data (especially abbreviated credit card numbers) to refer to the specific payment later. Due to this procedure in place one can conclude, that breaking into the systems of a merchant will not result in any leaked credit card information, if the merchant follows these guidelines. \\

The \textbf{issuer} is a valuable target for hacking into the backend systems with the objective to leak a massive amount of credit card and owner information. As a financial institute the issuer have to follow a huge set of regulations and safety procedures to be able to participate on the market. It can be assumed that at least the same safety mechanisms are valid as are in place for the \gls{PSP}. This means constantly monitoring the internal systems with an intrusion detection mechanism and blacklisting any leaked credit card. In addition to the monitoring of all online activities (as also the \gls{PSP} does) the issuing bank can monitor activities done with the credit card in the offline world too. In case of suspicious activities the credit card can be blocked and a new one will be send out to the credit card owner. \\

The \textbf{consumer} is also a valuable target for eavesdropping on credit card and personal information. She is also the weakest and most unsecure party in the whole E-commerce scenario. As said above a lot of the protection mechanisms of the other participants are relying on implementing industry standards and on constantly monitoring the own systems for malicious activities. This can not be securely said about the computer of the consumer. Whether she is using up-to-date security programs (e.g.\ an Antivirus tool and a firewall) on her computer or not is out of reach of the others to verify. Additionally, a consumer can fall victim to a phishing attack, that will send her to a malicious Web site with the intend to get her personal information. In some seldom cases the consumer might cooperate with the fraudster, or might be the fraudster herself with the intend to trick the system for her self-interest. Due to this the E-commerce fraud investigation can not rely on information from the consumer, but instead has to figure out if the transaction in question was made from the real owner of the credit card or from a frauster.

% subsection leaking_credit_cards (end)

\subsection{E-commerce fraud strategies}
\label{subsec:strategies_fraudster}

After a fraudster has got access to leaked credit card information she can come up with the following strategies to trick the E-commerce system:\@

\begin{enumerate}
  \item a fraudster owns \textbf{one} leaked credit card information and try to use it for ordering products from \textbf{multiple} merchants on the Internet
  \item a fraudster owns \textbf{multiple} leaked credit card information and try to use them for ordering products from \textbf{one} merchant on the Internet
  \item a combination of the two cases mentioned before, that can also be related to as a series of the first fraud activity
\end{enumerate}

In the \textbf{first scenario}, in which the fraudster is trying out a leaked credit card for ordering products on Web shops of various merchants, each of the merchant only sees the transaction that takes place in her system. It will make it more difficult for the merchant to detect whether this is a fraud transaction or not, because she is not aware of the attempts the fraudster did on other merchant's Web shops. \\

As each merchant will rely on a \gls{PSP} or issuer to verify the credit card payment, it is in the responsibility of these participants to recognize fraud transactions in this scenario. To be able to do so, the \gls{PSP} and also the issuer are monitoring the usage of credit cards and are actively looking for suspicious activities. The fraud prevention mechanisms in place are mostly working on a rule-based, and in some cases also score-based systems running in the internal network of the \gls{PSP} and issuer. These systems are fed with the information the merchant sends with the payment authorization request and will come up with either:\@

\begin{enumerate}
  \item Yes, this looks like a fraudulent transaction and has to be blocked
  \item No, this seems to be a valid transaction and should be acknowledged
  \item Maybe, this transaction might be valid, but there is some uncertainty in the validation of it. These edge cases are routed to a human operator of the \gls{PSP} or issuer to decide on how to proceed with it
\end{enumerate}

As a recent study shows the success rate of the fraud prevention systems heavily relies on the techniques used to validate the transaction data \citep{rana2015survey}. The outcome is, that ca. 70 to 80 percent of the fraudulent transactions will be recognized as such and blocked successfully. That still means 20 to 30 percent of fraudulent transactions could not be recognized as such. For handling these cases the organisations employ special trained staff, that is operating 24/7 and 365 days a year, to be able to manage these edge cases. \\

As stated in the introduction of this Master thesis, there is a shift from the offline credit card fraud to the online world. This is also resembled in current figures of E-commerce fraud incidents, whose show that it makes up to 85 percent of all credit card fraud attempts and have on average a transaction value of 500 to 600 EURO.\\

As the \gls{PSP}s and the issuers do not have any order details, they can only decide on the information given during the authorization request (see Section~\ref{sec:stakeholder_analysis}). At most they can validate the branch the merchant is operating in, and it might come as no surprise that the fraudsters are regularly using Web shops of merchants, who offer either electronics, clothings, entertainment- or travel-related products. These are also the most commonly used sources of \textbf{\underline{valid}} E-commerce transactions, and will therefore make any fraudulent transaction very difficult to detect. \\

At the end it might be the owner of the credit card, who detects suspicious activities on her credit card account and informs the issuing bank about it. Based on current regulations and laws the issuing bank has to rollback the fraudulent transaction on request of the consumer, which means that the merchant will have to cover the costs of the E-commerce fraud (as she is not receiving the money for the products that has been already shipped to the fraudster). \\

Looking at the \textbf{second scenario} of the E-commerce fraud strategies at the beginning of this section, a merchant will receive multiple requests from a fraudster, who is trying out various leaked credit cards for finishing an order. This kind of E-commerce fraud can be recognized at the systems of the merchant due to the same source \gls{IP} address of the requests, or due to having the same shipping address for orders with different credit cards. Therefore, one can conclude that also merchants must take an active role in the fraud prevention process (if they do not already do so) and try to minimize the amount of fraudulent transactions taken place on their Web shops. As this scenario is likely be manageable with additional fraud prevention mechanisms at the merchant, and does not need to involve other parties of the E-commerce scenario to figure out the validity of the transaction, this second scenario falls out of scope of this Master thesis.

% subsection strategies_fraudster (end)

\subsection{E-commerce fraud incidents handling}
\label{subsec:e_commerce_fraud_handling}

If the fraud prevention systems at the \gls{PSP} or issuer are detecting a suspecious transaction, an operator working in a special department within the organisation will be informed about the transaction via a notification on his computer. This operator will have to decide whether the transaction looks valid and should be acknowledged, or seems to be fraudulent and has to be denied. To be able to decide this, she is going to look into the recent usages of the credit card in question. Whereas it will be easy to recognize that a credit card, that was just being used in a shop in Germany, could not be used in a shop in US or Asia within a short timeframe due to physical constraints in the real world, the same consumer can order products from an US or Asian online retailer with ease within minutes. So these initial geographical contraints, that work well with real world usage patterns of credit cards (a fraud prevention mechanism called geo-fencing), will no longer work well in the E-commerce scenario. \\

So the operator has to found her decision on the transaction information at hand. Initially she can check for the amount that has been paid with the credit card. One can assume that small amounts will be covered by the \gls{PSP} or issuer, who will take over the risk for a false authorization. But with an increased value of the items ordered, the \gls{PSP} or issuer is putting back the risk to the merchant in case of customer complaints later. At a second glance the operator can verify whether the consumer has had any business relationship with the merchant in the past or not as well as verify the retail branch the merchant operates in. Although these are weak hints for investigating the validity of an E-commerce transaction as they can be bypassed by the fraudster with ease (see above). \\

To make a solid decision the operator will have to get in contact with the merchants the credit card has been used with recently, and have to ask for additional information such as:\@

\begin{itemize}
  \item does the consumer owns an user account with the Web shop?
  \item what is the consumer usually looking for in the Web shop?
  \item does the shipping address matches the billing address for the order?
  \item if not, has the user send products to this shipping address in the past?
  \item what has been ordered, incl.\ detailed product information such as brand, model, product categories, \ldots
\end{itemize}

In some cases the \gls{PSP} or issuer might have had a business relationship with the online merchant in the past. So the \gls{PSP} or issuer might already know whom to contact from the support personnel of the merchant. But in most cases the contact person might not be known to the \gls{PSP} or issuer resulting in sending a request to the general support staff via the contact formulars on the merchants Web site. \\

Getting the right information might take time as the correct addressee from the support department of the merchant is unknown, the merchant do not have specialized staff at hand to handle these kind of issues, or there are misunderstandings on handling the case due to language barriers or different incentives between the participants. Therefore one can assume that this detailed analysis of any suspecious transaction will not take place today, instead most of these transactions will be acknowledged without doubt after a first short look. \\

Still the merchants as well as the \gls{PSP} and issuer have a high incentive for keeping the success rates and numbers of any fraudulent activity low. For the \gls{PSP} and issuing bank there are regulations stating that at maximum only 1 thousands of the overall transactions should be fraudulent (note: numbers stated are valid for the EU). This keeps the pressure on these financial instituts to invest in fraud prevention techniques for being able to stay in business. For the merchant it is of high interest that a fraudulent transaction can be resolved, before the fraudster receives the ordered products. In the worst case scenario just one successfully performed fraudulent transaction in an E-commerce shop will trigger hundreds if not thousands of following attempts from other fraudsters, as past experiences have shown. \\

% subsection e_commerce_fraud_handling

% section scenario_fraud (end)
