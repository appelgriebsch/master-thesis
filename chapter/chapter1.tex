%!TEX root = ../TemplateMasterThesis.tex

\chapter{Chapter One} % (fold)
\label{cha:chapter_one}

Introduction of chapter, overview, ...


\section{Section about tables} % (fold)
\label{sec:section_about_tables}

In this section, an example for a table is given:

\begin{table}[ht]
	\centering
	\caption[Short title Exampletable]{Here you can insert a longer \textbf{HEADER} for the table}
		\vspace{1.0em}
		\begin{tabular}{|l|r|}
\hline
Text & 12\% \\
\hline
Text & 34\% \\
\hline
Text & 56\% \\
\hline
Text & 78\% \\
\hline
Text & 90\% \\
\hline
		\end{tabular}
	\label{tab:tableexample}
\end{table}

% TODO don't forget  to check


These are just examples - tables can have very different appearances. You can play along with tables - the following might be an alternative:

\begin{table}[ht]
	\centering
	\caption[Alternative Table]{Another table with coloured headline}
		\vspace{1.0em}	
	\begin{tabular}{|c|l|}
		\hline
		\rowcolor[gray]{0.9}\textbf{numbers} & \textbf{text} \\
		\hline
		\hline
		1 & This text flush-left \\
		\hline
		2 & while the numbers are \\
		\hline
		3 & centred \\
		\hline
	\end{tabular}
	\label{tab:tablealternative}
\end{table}

Please note, that the labels of tables appear above while labels of figures appear underneath. \\

The creation of tables can be exhausting, you might make use of this helpful tools:
\begin{itemize}
	\item Excel to \LaTeX{} Converter\\ \url{http://www.heise.de/download/excel2latex.html}
	\item Apple Script: Numbers to \LaTeX{} \\ \url{https://gist.github.com/pgundlach/386384}
	\item Gnumeric (has an export function for \LaTeX{}): \\ \url{https://projects.gnome.org/gnumeric/}
	\item OpenOffice, Calc2LaTeX: \url{http://extensions.openoffice.org/de/project/calc2latex-macro-converting-openofficeorg-calc-spreadsheets-latex-tables}
\end{itemize}

% section section_about_tables (end)


\section{Section in which the tables are referred to} % (fold)
\label{sec:section_in_which_the_tables_are_referred_to}

In this section, the afore mentioned tables are referred to within the text. This is a reference for the first table \ref{tab:tableexample}, and here the reference is to the alternative table \ref{tab:tablealternative}
Note, that you should give every table, figure, section, chapter, etc. an individual label, so that you can use this for reference if needed.

% section section_in_which_the_tables_are_referred_to (end)


\section{Section about quotation marks} % (fold)
\label{sec:section_about_quotation_marks}


You have different option to create quotation marks. Since a package is included here you can make it like this:
\begin{itemize}
	\item \enquote{Using the package: csquotes}
	\item Or like this ``example''
	\item Or like this "example". \\ You might notice that this alternative quotation marks look differently!
\end{itemize}


% section section_about_quotation_marks (end)


% chapter chapter_one (end)
