%!TEX root = ../TemplateMasterThesis.tex

\chapter{Quotations and the use of Bibtex} % (fold)
\label{cha:quotations_and_the_use_of_bibtex}

In \LaTeX only those literature and sources will appear in the bibliography, which have been referred to within the text. So even if you insert all literature in the \texttt{references.bib} file, the appear not until you referred to them in the text. This is not a bug - it's a feature, since it is common practice in academia, that you don't list sources in the literature you didn't use. It's therefore very convenient, that you don't have to check your document, after for example deleting some parts which included references. If those would have been the only ones, the literature will automatically not appear in the bibliography. \\

Creating the \texttt{references.bib} file can be exhausting, but you can make use of several tools. 

\begin{itemize}
	\item BibTeX Converter by Amazon: \url{http://lead.to/amazon/en/}
	\item BibTeX Converter by Springer: \url{http://sydney.edu.au/engineering/it/~niu/cgi-bin/springer.cgi}
\end{itemize}

Digital libraries also often have a feature to export formats for BibTeX, for example

\begin{itemize}
	\item ACM digital library: \url{https://dl.acm.org/}
	\item Google Scholar: \url{https://scholar.google.com/}
	\item IEEE Xplore: \url{http://ieeexplore.ieee.org}
\end{itemize}

Most reference management software also have a function for exporting a bibtex file, for example:

\begin{itemize}
	\item Papers (Mac, PC, iOS), $\sim$35 Euro student discount: \url{http://www.papersapp.com/}
	\item Mendeley (Windows, Mac, Linux, iOS): \url{http://www.mendeley.com/}
\end{itemize}

However you should check the bibtex code immediately, because sometimes some informations might be missing. Be careful if you selected the correct edition / publisher etc.




% chapter quotations_and_the_use_of_bibtex (end)