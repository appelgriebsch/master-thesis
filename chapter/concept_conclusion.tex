%!TEX root = ../MasterThesis.tex

\section{Conclusion}
\label{sec:concept_conclusion}

As the previous section showed, existing approaches are of limited use for the design of a collaborative system to support the \gls{E-commerce} fraud investigation scenario described in Section~\ref{sec:scope_thesis}. The leading approach for such a system will have to combine the best characteristics from the Web Service and the Semantic Web designs. \\

As for the Web Service approach, the most valuable aspects of it are: \@

\begin{itemize}
	\item access to the \gls{HTTP} endpoints can be limited to a certain set of communication partners
	\item these partners have to authenticate with each Web Service first
	\item based on the identification of the partners only certain parts of information can be returned, and execution of operations can be restricted
\end{itemize}

Looking at the Semantic Web approach, it's most interesting functionalities are: \@

\begin{itemize}
	\item providing information in a semantically self-contained way
	\item the ability to merge information from different \gls{RDF} data stores locally
	\item the graph-based data model underlying the \gls{RDF} data stores
	\item the usage of \gls{SPARQL} to query and analyze the locally combined data sets
\end{itemize}

In the following Chapter~\ref{cha:system_design} the thesis will come up with an approach, that uses the fundamental technologies from the Semantic Web for information sharing and integration as well as peer-to-peer communication technologies for securing and restricting access to the \gls{RDF} data sets from the relevant participants of the \gls{E-commerce} fraud investigation scenario. 

% section conclusion (end)
