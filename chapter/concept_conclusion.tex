%!TEX root = ../MasterThesis.tex

\section{Conclusion}
\label{sec:concept_conclusion}

To support the investigation of \gls{E-commerce} frauds as described in Section~\ref{sec:scope_thesis} the collaborative system has to collect and combine transactional information from various online merchants of Web shops a credit card has been used with recently. The system has to support the combination and linking of the transactional details by utilizing a graph-based data model. Doing so will allow the system to classify and cluster the transaction information based on various criteria, which can help the investigator to figure out abnormal behavioural patterns in the credit card usage on the Internet. Visualizing the combined data set can make use of the graph-based data model and present the transaction details as a clustered graph on screen. Additionally, the representation of the information can change based on the requirements of the investigators. \\

As the previous section showed in detail existing approaches are of limited use for the collection and combination of dispersed transactional details in this scenario. The leading approach for the \gls{E-commerce} fraud investigation system will have to combine the best characteristics from the Web Service and the Semantic Web designs. \\

As for the Web Service approach the most valuable aspects of it are: \@

\begin{itemize}
	\item access to the \gls{HTTP} endpoints can be limited to a certain set of communication partners,
	\item these partners have to authenticate with each Web Service first,
	\item based on the identification of the partners only certain aspects of the information can be exchanged, and execution of Web Service operations can be restricted.
\end{itemize}

When looking at the Semantic Web approach it's most beneficial functionalities are: \@

\begin{itemize}
	\item providing information in a semantically self-contained way,
	\item the ability to merge and link together information from different \gls{RDF} data stores locally,
	\item the graph-based data model underlying the \gls{RDF} data stores,
	\item the usage of \gls{SPARQL} to query and analyse the locally combined data set.
\end{itemize}

In the following Chapter~\ref{cha:system_design} the Master Thesis will come up with an approach that uses the fundamental technologies from the Semantic Web for information sharing and integration as well as peer-to-peer communication technologies for securing and restricting access to the \gls{RDF} data sets for relevant participants of the \gls{E-commerce} fraud investigation scenario only.

% section conclusion (end)
