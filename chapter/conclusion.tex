%!TEX root = ../MasterThesis.tex

\chapter{Conclusion and Future Work} % (fold)
\label{cha:conclusion}

\section{Conclusion}
\label{sec:conclusion}

Initially, this Master Thesis started with three fundamental assumptions: \@

\begin{enumerate}
	\item the use of an insecure computer network to transfer credit card information in the \gls{E-commerce} scenario make them always subject to frauds and counterfeits,
	\item the existing technological solutions to detect and prevent them will not be able to cover 100\% due to the complexity and dynamics of the whole system, and
	\item the substantial analysis of any suspicious transaction is not taken place today due to the enormous effort and time needed to collect all required information manually
\end{enumerate}

As this Master Thesis has pointed out in Chapter~\ref{cha:introduction} and Chapter~\ref{cha:context_analysis}, the first and second premise has been seen a lot of research and development in recent years. But neither an industry-wide usage of the \gls{PCI/DSS} standards to securely process personal and payment-related information, nor the wide adoption of rule-based or score-based fraud prevention systems has been able to stop deceivers from cheating the \gls{E-commerce} system, and bringing harm to the merchants and consumers alike. Whereas the Master Thesis has shown the impact of \gls{E-commerce} frauds to the merchants, which are usually the ones that have to cover the losses, it also makes clear that any successful fraud attempt will reduce the thrust into the \gls{E-commerce} system that is a substantial part of the global economy already. \\

Thus, keeping the amount of fraudulent transactions down can not be achieved with technology alone, but require a sharing of information between experts coming from various organizations. As the Master Thesis explained in Section~\ref{subsec:e_commerce_fraud_handling}, this consolidation of information about a suspicious transaction is currently not being done in-depth, as this would require a lengthy manual communication and analysing process by the investigator of an issuer or \gls{PSP}. This fact has lead to the third premise, and eventually to the hypothesis that the introduction of a collaborative system, which makes use of Semantic Web and peer-to-peer communication technologies to collect the information of the relevant stakeholders and link them together, can improve this situation significantly. \\

As explained in Section~\ref{sec:system_approaches}, the existing approaches for integrating information from different sources are either to restrictive to work with on a large scale (Web Service), or are way to open for the sharing of personal and payment-related information (Semantic Web). Therefore, a new proposal for a collaborative system has been developed in Chapter~\ref{cha:system_concept} and Chapter~\ref{cha:system_design}, which uses suitable aspects from the Semantic Web standards, as well as a secure \gls{P2P} communication network between the stakeholders for collaborating on \gls{E-commerce} fraud incidents. \\

Due to the fact that the information will be provided from dispersed organizations, the proposed solution have to deal with differences in structure and wording of them. Under these circumstances, the \gls{W3C} standards such as \gls{RDF}, \gls{RDFS}, \gls{OWL} and \gls{SPARQL} show their full potential. As part of the Semantic Web initiative they already solve these issues on a global scale. However, those standards only provide the basics for integrating distributed information. Additional steps are required to build intelligent applications on top of them. In Chapter~\ref{cha:system_concept} the Master Thesis showed that the information has to be combined into a graph-oriented representation of the \gls{E-commerce} activities done with a credit card recently. Based on the initial clustering of order details by merchant, a subsequent mapping and linking of the information has to be achieved to be able to cluster the transactions on different aspects with the objective to find abnormal activities. \\

As Chapter~\ref{cha:system_design} explained, the Semantic Web standards provide various axioms in the \gls{RDFS} and \gls{OWL} specifications to support the mapping of information coming from different sources. Based on these predicates the \gls{RDF} data store (especially the reasoner in it) can \emph{infer} additional attributes (aka relations) into the combined data set. These mapping expressions can be either managed and injected by the participant, who is doing the merging of the information, or can be integral part of the shared \gls{RDF} data sets already. Particularly the latter option will ease the integration of different data sources, but requires the participants to look into commonly used \gls{RDF} vocabularies or ontologies. A selection of the available vocabularies, which might fit the \gls{E-commerce} fraud investigation, was also depicted in this chapter. In addition to the mapping of the information, certain aspects of the transactions have to be linked together to be able to cluster and analyse them in detail. Linking information on the Semantic Web relies on either using the same unique \gls{URI}s for identical objects in different data sets, or on inferring equality based on property constraints set in the schema definitions. A detailed explanation of these options can also be found in Chapter~\ref{cha:system_design}. \\

At the end the proposed solution was using a partially centralized \gls{P2P} architecture that has the issuer of a credit card at its centre. This decision was taken, because the issuers are those stakeholders, who will figure out about suspicious activities on a credit card first. They will also have to decide whether a transaction is fraudulent or not. In some edge cases, in which the fraud prevention system is unsure about the status of a transaction, an investigator of the issuer will initiate a collaborative session with affected participants. Those parties will share their relevant information with the issuer, and collaborate on the investigation of the incident. Due to the fact that the issuer as a financial institution has to follow strict regulations and data protection laws, there should be no serious problems for sharing detailed order information with them. However, some of the information of the stakeholders might be critical and have to be obfuscated in some way. Available methods to do so are part of the proposed collaborative system design. \\

Thus, this Master Thesis has shown that the technical issues, which lead to fraudulent transactions in the \gls{E-commerce} scenario, could not be solved completely by introducing new security procedures, but require a collaboration of different experts on the edge cases. This collaboration can become more efficient and effective, if a shared information space is used, which combine and link together the information from the relevant stakeholders. The existing Semantic Web and \gls{P2P} communication technologies can act as enablers to design and develop such as collaborative system.

\section{Outlook: Towards a decentralized \gls{P2P} system}
\label{sec:p2p_decentralized_system}

One major concern of the proposed solution can be the partially centralized \gls{P2P} architecture of the collaborative system, which results in duplicating the relevant information from any stakeholder to the issuer of a credit card in question. Even though the solution offers ways to obfuscate critical information, a \emph{decentralized} \gls{P2P} architecture for the collaborative system will likely resolve any complains about potential privacy issues. In such an architectural approach each node is equal and keeps their local data. This will have a huge impact on the way the fraud investigation is done. The combined data set will no longer be available for analysing on a single data store with \gls{SPARQL}, but has to be done in a distributed way as well. Thus, the decentralized \gls{P2P} system will require new methods for querying, accessing and linking the information coming from different stakeholders. Further research on available protocols and procedures is needed to examine the feasibility of such a distributed \gls{P2P} system architecture.

% sec p2p_decentralized_system

% chapter conclusion (end)
