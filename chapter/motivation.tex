%!TEX root = ../MasterThesis.tex

\section{Motivation}
\label{sec:motivation}

\begin{quotation}
    \textit{\enquote{When it comes to fraud, 2015 is likely among the riskiest season retailers have ever seen, […]
    it is critical that they prepare for a significant uptick in fraud, particularly within e-commerce channels.}}
\end{quotation}
This statement from Mike Braatz, senior vice president of Payment Risk Management, ACI Worldwide in \citep{Reuters2015}
shows the dramatic shift in credit card fraud from the offline to the online world, that retailers are starting to face
nowadays. \\

In general credit card fraud can occur if a consumer has lost her credit card or if the credit card has been stolen
by a criminal. This usually results in an \textbf{identity theft} by the criminal, who is using the original credit card
to make financial transactions by pretending to be the owner of the card. Additionally, a consumer might hand over her
credit card information to an untrustworthy individual, who might use this information for her own benefit.
In the real world scenario there is usually a face-to-face interaction between both parties.
The consumer, wanting to do business with a merchant or interacting with an employee of a larger business, has to hand over
her credit card information explicitly and can deny doing so if she faces a suspicious situation.
The criminal on the other hand must get access to the physical credit card first, before she is able to make an
illegal copy of it --- a process called \textbf{skimming}. The devices used to read out and duplicate the credit card
information are therefore called skimmers. These can be special terminals, that the criminal uses to make copies of
credit cards she gets her hands on, or they can be installed in or attached to terminals the consumer interacts with on her own
\citep{ConsumerAction2009}. All of these so-called \textit{card-present transaction} scenarios have seen a lot of improvements in security
over the last years. Especially the transition from magnetic swipe readers to EMV chip-based credit cards makes it more difficult
for criminals to counterfeit them \citep{Lewis2015}. \\

As of this criminals are turning away from these card-present transaction scenarios in the offline world. Instead they are focusing on transactions
in the online and mobile world, in which it is easy to pretend to own a certain credit card. Most online transactions (either e-commerce or m-commerce)
rely \textbf{\textit{only}} on credit card information like card number, card holder and security code for the card validation process – as of this these interactions
are usually called \textit{card-not-present transactions}. This credit card information can be obtained by a criminal in a number of ways.
First she might send out \textbf{phishing emails} to consumers. These emails mimic the look-and-feel of emails from a merchant or bank, that the consumers are normally
interacting with, but instead navigating the consumers to a malicious web site with the intend to capture credit card or other personal information \citep{ConsumerAction2009}.
Additionally, criminals can \textbf{break into the web sites} of large Internet businesses with the goal of getting access to the underlying database of customer information,
that in most cases also hold credit card data \citep{Holmes2015}. Additionally, some of the online retailers are not encrypting the transaction information before transmitting
them over the Internet; a hacker can easily start a \textbf{man-in-the-middle attack} to trace these data packages and get access to credit card and/or personal information
in this way \citep{Captain2015}. \\

Based on this it should come not as a surprise that the growth rate of online fraud has been 163\% in 2015 alone \citep{PYMNTS2016}.
This results in huge losses for the global economy every year and it is expected that retailers are losing \$3.08 for every dollar in fraud incurred in 2014
(incl.\ the costs for handling fraudulent transactions) \citep{Rampton2015}. These fraudulent transactions also impact the revenue of the online retailers.
Here we have seen a growth of 94\% in revenue lost in 2015. Overall it is estimated that credit card fault results in \$16 billion losses globally in 2014 \citep{PYMNTS2016}
\citep{BusinessWire2015}. \\

While it is possible to prevent fraudulent transactions in the card-present real-world scenario (mainly due to introducing better technology and establishing organizational
countermeasures in the recent past), it is more difficult to do so in the card-not-present online- and mobile commerce scenarios, which are lacking face-to-face interactions
and enable massive scalability of misusing credit card information in even shorter time frames \citep{Lewis2015}. Large online retailers have tried to establish countermeasures
and transaction data analysis technologies to lower the rate of fraudulent transactions to a manageable amount. But this is still an expensive and inefficient solution to integrate
into the retailers’ business processes, and is largely driven by machine-learning techniques and manual review processes \citep{Brachmann2015}. Additionally, it can be assumed,
that the online retailers are getting into a Red Queen race with the criminals here: with every new technology or method introduced they might just be able to safe the status quo.
This is largely due to the facts, that there will be no 100\% security for such a complex and interconnected system like an e-commerce or m-commerce shop, the criminals will also
increase their efforts and technology skills to adapt to new security features and most importantly retailers will always have to make a trade-off between the \textit{performance}
of the transaction processing, the \textit{usability} of the web shop and the overall \textit{security} of it.

% section motivation (end)
