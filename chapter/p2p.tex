%!TEX root = ../MasterThesis.tex

\section{Peer-to-peer communication}
\label{sec:p2p_communication}

\subsection{Centralized vs. Decentralized Web Architectures}

- in a classical client-server scenario a single server is storing information and distributing it to the clients \\
- the information is centralized and under control of the provider \\

- a P2P network considers all nodes equal \\
- each node can provide information to any other node \\
- information in a P2P network has to be indexed so that the correct node is queried for it \\
- the index itself has to be stored somewhere (e.g. on a central server like Napster or in a distributed manner spread over the nodes of the P2P network) \\
\\
- a P2P system has an high degree of decentralization \\
- the system is usually self-organizing (adding new or removing members automatically) \\
- the whole system is usually not controlled by a single organisation and spread over various domains \\
- it tends to be more resilent to faults and attacks \\
- can be used for file \& data sharing, media streaming, telephony, volunteer computing and much more \\
\\
- can be categorized by the degree of centralization into: \\
  1) partly centralized P2P systems (have a dedicated controller node that maintains the set of participating nodes and controls the system) \\
  2) decentralized P2P systems (there are no dedicated nodes that are critical for the system operation) \\
\\
\label{sec:central_decentral_arch}

% section central_decentral_arch (end)

\subsection{Initiating a communication session}
\label{sec:p2p_init_session}

- depends on the structure of the P2P system
- in a partly centralized P2P system new nodes join the network by connecting to the central controller (wellknown IP address) \\
- in a decentralized P2P system new nodes are expected to obtain via a separate channel the IP address to connect to (usually a bootstrap node that helps to set up the new node) \\
\\

% section p2p_init_session (end)

\subsection{Finding communication peers}
\label{sec:p2p_finding_peers}

- also known as the overlay network in a P2P system \\
- can be represented as a directed graph containing the nodes and communication links between them \\
- can be differentiated between unstructured and structured overlays \\
- unstructured overlay networks have no constraints for the links between nodes; therefore the network has no particular structure \\
- structured overlay networks assign an unique identifier from a numeric keyspace to each node; these keys are used to assign certain responsibilities to nodes
on the network; as of this routing can be handled more efficiently \\
- in partly centralized P2P systems the controller is responsible for the overlay formation \\
\\
- in partly centralized P2P system an object is typically stored at the node that inserted the object \\
- the central controller holds the information about which objects exist and which nodes hold them \\
\\
- in unstructured systems the information is typically stored on the nodes that introduces them \\
- to locate an object a query request is typically broadcasted through the overlay network \\
- often the scope of the request (e.g. the maximum number of hops from the querying node forward) is limited to reduce the overhead on the system \\
\\
- in structured systems a distributed index is maintained in the form of a distributed hash table \\
- this DHT holds the hash value of the (index) key and the address of the node that stores the value \\
\\

% section p2p_finding_peers

\subsection{Transmitting Data}
\label{sec:p2p_data_transmit}


% section p2p_data_transmit (end)

\subsection{Available Protocols}
\label{sec:p2p_protocols}

% section p2p_protocols (end)

% section p2p_communication (end)
