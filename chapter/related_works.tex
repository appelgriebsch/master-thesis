%!TEX root = ../MasterThesis.tex

\chapter{Related Works}
\label{cha:related_works}

Gayatri R. Ankhule et al. give an overview of \gls{E-commerce} and specify the different types of it. In addition to that the authors also show the positive and negative impact of the current implementation of \gls{E-commerce} for organizations, individuals and society in general. After demonstrating the advantages of it (and the reasons for the growing success) they assert that the safest way to checkout an order on a Web shop is by using a credit card, because in the case of fraud attempts on a credit card consumers can rely on regulations and protection laws in place to rollback malicious transactions. At the end of the paper they conclude that a successful \gls{E-commerce} shop is not just a technical problem, but affects the whole business operation \citep{ankhule2015overview}. \\

The paper from Sobko discusses what non-cash transactions are and shows ways how fraudsters try cheat the system. It starts with a classification of non-cash payments including credit and debit cards that are handed out by financial institutions to individuals for doing commerce. The authors also note the ways to trick an individual with the objective to get access to credit card information such as phishing and skimming. They notice that once a transaction has been successfully executed with a stolen credit card the information about it will be sold on the black market to other fraudsters, who will then use the same credit card to make larger purchases. Further on the paper discusses the impact of fraudulent transactions on the merchants and credit card owners, as well as discusses technological advances and legislations that have been developed to protect against non-cash frauds \citep{sobko2014fraud}. \\

The study of Pritikana Sen et al. starts with an introduction to the subject of \gls{E-commerce} and iterate on the classification of it. It states the benefits of \gls{E-commerce} (e.g.\ the global reach of Web shops) as well as its limitations. Here it mentioned explicitly the security of the system and the communication protocols used. The paper lists the relevant stakeholders of an \gls{E-commerce} transaction and describes the credit card payment process. It concludes with an analysis of the security features of a Web shop and shows that those are not limited to technical aspects alone, but always include the consumer and his behavior on the Internet \citep{sen2015study}. \\

The research of Priya J. Rana et al. shows possible frauds in \gls{E-commerce} and how they can be detected with current fraud prevention systems. The show different implementations of fraud detection algorithms that range from simple rule-based filtering to score-based solution using fuzzy logic. They conclude that general systems in use can cover up to 80\% of fraudulent transactions at manageable efforts and costs. More coverage can be achieved by combining existing solutions with information of the card owners profile, which introduces credit card usages patterns into the analysis. Still this solution is very expensive to implement and operate \citep{rana2015survey}. \\

The paper from Carvalho et al. looks into the financial crime investigation process by using banking frauds as example. It shows that the investigation is a complex task that needs further collaboration between experts. Still just sharing the information will not be enough as a common understanding of the different aspects and terms is required. Therefore they state that finding a common language to exchange information is very important for the success of the investigation. Based on this finding the paper also develops an ontology to describe the domain of banking fraud investigation. It elaborates on the objects and their relations in detail and shows that reusing concepts and terms from existing vocabularies can be helpful when designing an own ontology. The paper concludes that semantic technologies can have a positive impact on the crime investigation as they are providing basic reasoning capabilities on the data sets as well as merging information from different sources. These features will allow a crime investigator from a law enforcement agency to inspect and analyze more complex information. Finally the paper considers semantic technology to be very important in future cybercrime inspection \citep{carvalhoapplying}. \\

- ``Linked data-the story so far'' \citep{bizer2009linked} \\
- ``Linked data-as-a-service: the semantic web redeployed'' \citep{rietveld2015linked}

- ``Goodrelations: An ontology for describing products and services offers on the web'' \citep{hepp2008goodrelations} \\
- ``Schema.org: Evolution of structured data on the web'' \citep{guha2016schema} \\
- ``Leveraging WebRTC for P2P content distribution in web browsers'' \citep{vogt2013leveraging} \\
- ``Content-centric user networks: WebRTC as a path to name-based publishing'' \citep{vogt2013content} \\
- ``RDFPeers: a scalable distributed RDF repository based on a structured peer-to-peer network'' \citep{cai2004rdfpeers}

% chapter related works (end)
