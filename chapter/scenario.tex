%!TEX root = ../MasterThesis.tex

\section{Scenario Description}
\label{sec:scenario_description}

- different forms of e-commerce exists: \\
1. Business-To-Business (B2B): electronic trading between companies for improving supply-chain processes \\
2. Business-To-Consumer (B2C): electronic trading between merchant and consumers \\
3. Consumer-To-Consumer (C2C): electronic trading between consumers - e.g. eBay \\
\\
- focus of this thesis is on B2C \\
- consumer is using an e-commerce shop of a merchant on the Internet \\
- she is ordering products from the catalog of the merchant and uses her credit card to pay the bill \\
- the merchant is relying on a third party service to handle the payment process (e.g. PayPal) \\
- this payment processor is routing the transaction amount due to the issuer of the credit card (e.g. a bank or credit card institute) \\
- the merchant have a business relationship with her own bank and uses the service to acquire the outstanding amounts from consumers \\
- in the clearing process the acquiring bank is settling any outstanding transaction with an issuing bank \\
\\
Workshop ErsteBank Wien: \\
- usually banks are monitoring the usage of credit / debit cards and are looking for suspicious activities \\
- this fraud prevention mechanism is working on a rule-based (non self-learning) or score-based (self-learning) software from a third party \\
- the outcome of the fraud prevention could be: yes (this is a fraudulent transaction, pls. block it), no (everything looks fine, pls. continue with the process) or maybe (uncertain, pls. let a human decide how to proceed) \\
- in case of the maybe result an alert is triggered to one of the support staff of the bank (operating 24/7/365) \\
- still this kind of fraud prevention mgmt. can not solve all issues due to the amount and frequency of the transactions, there is generally a
fraud-to-sales ratio of max. 0.11 percent in the EU (meaning 1 promille of transactions are fraudulent) \\
- still the success rate of fraud prevention is rougly 70-80 percent, means of the fraudulent transactions nearly 80 percent are blocked correctly \\
- for the rest: the consumer has to actively trigger an investigation, if the case is valid usually the issuing bank will cover the cost (in case of larger amounts an insurance will take over). \\
- the bank is responsible for pushing the consumer to file a case at police \\
- most of the filed cases could not be resolved \\
- there is no court decision yet in which circumstances the consumer might be guilty as well \\
\\
- ca. 85 percent of frauds are e-commerce frauds (EU: 70-90 percent). Hotspots are Germany, France and US. Frauds are coming from Travel-Shops or Online Merchants and the amount is on average between 500-600 EUR \\
- e-Commerce frauds will usually not filed at police; in most cases the acquirer is in charge to handle the issue \\
- if it is known that a merchant has been hacked the bank is usually issuing new credit cards to all affected consumers automatically \\
- otherwise the bank has to get in contact with the acquirer and/or merchant to figure out if the transaction is fraudulent \\
- usually the banks only have credit card related information from a consumer (no detailed information about the ongoing transaction), whereas the merchant and the acquirer have the detailed records of the order at hand \\
- various regulations make it hard to shre detailed information with involved parties (even if they have special agreements signed between them) \\
- main questions for e-commerce fraud: who is the party that is the victim of the incident? Is it really a fraudulent transaction? \\
- e-commerce fraud can not be handled by technology alone, at best the fraudulent transaction can be blocked on the merchant side (due to the information given by the consumer like items, prices, delivery address, ...) \\
- in the worst case one successful fraudulent transaction in an e-commerce shop will trigger hundred and thousands of attempts -> so the awareness for the issue has to be at merchant side \\
- at the end: much effort is assumed to bring all the experts together and solve the issue by putting their individual know-how on the table \\


% section scenario description (end)
