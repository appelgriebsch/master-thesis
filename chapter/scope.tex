%!TEX root = ../MasterThesis.tex

\section{Scope of this Master Thesis}
\label{sec:scope_thesis}

As laid out in the previous section the most interesting E-commerce fraud scenario is the one, in which a fraudster uses a leaked credit card information to order products or services from various merchants on the Internet. This is currently most likely to be successful as there is a lack of information on the side of the merchant as well as the \gls{PSP} or issuer. Each of the affected merchants just noticed the single transaction that takes place on her Web shop, without knowing about the other attempts the fraudster do on other Web shops. The \gls{PSP} and issuer will notice the active use of the credit card on different Web shops though, but do not have any information about the transaction details. Therefore they could not correlate these information to check for suspicious activities. \\

Based on the current strategies of the fraudsters, whose will use the leaked credit card information in commonly used Web shops that deal with electronics, entertainment or travel-related products and services, it is more likely that these fraudulent transactions will not be recognized on time by the existing fraud prevention mechanisms in place. \\

A simple approach to solve this issue would be to just share more information of the ongoing transaction between the merchants, the \gls{PSP}s and the issuers. This might fail due to the restrictions and regulations for sharing personal related information, that each party has to follow though. Additionally adapting and harmonizing the communication interfaces between the Web shops from various online merchants and different \gls{PSP}s and issuers are an enormous undertaking and will likely fail due to different notion of the communication pattern and exchanged data structures between all affected parties. \\

This is the scenario this Master thesis will look into in detail and try to solve the most important question: is this really a fraudulent transaction? \\

Looking into the stakeholders, that can provide useful information to solve this question, one will come up with:\@

\begin{enumerate}
    \item \textbf{merchant}: providing additional information of the E-commerce transaction in question
    \item \textbf{\gls{PSP}/issuer}: providing additional information of the credit card usage and owner
    \item \textbf{\gls{ISP}}: can give hints whether a consumer has fallen victim to a phishing attack based on Internet access logs
    \item \textbf{\gls{LSP}}: provide information if the order has already been shipped or can be still halted
\end{enumerate}

% section scope_thesis (end)
