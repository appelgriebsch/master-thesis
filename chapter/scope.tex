%!TEX root = ../MasterThesis.tex

\section{Scope of this Master Thesis}
\label{sec:scope_thesis}

As laid out in the previous section, the most interesting E-commerce fraud scenario is the one, in which a fraudster uses a leaked credit card information to order products or services from various merchants on the Internet. This is currently most likely to be successful, because there is a lack of information on the side of the merchant as well as the \gls{PSP} or issuer. Each of the affected merchants just noticed the single transaction that takes place on her own Web shop, without knowing about the other attempts the fraudster do on other Web shops. The \gls{PSP} and issuer will notice the active use of the credit card on different Web shops though, but do not have any information about the transaction details. Therefore they could not correlate these information to check for suspicious activities. \\

Based on the current credit card usage patterns of the fraudsters, whose will use a leaked credit card information in commonly used Web shops, which deal with electronics, entertainment or travel-related products and services, it is more likely that these fraudulent transactions will not be recognized on time by the existing fraud prevention mechanisms in place today. \\

A simple approach to solve this issue would be to just share more information of the ongoing transaction between the merchants, the \gls{PSP}s and the issuers. This might be subject to fail though, because each party has to follow the restrictions and regulations for sharing personal-related information. Additionally adapting and harmonizing the communication interfaces between the Web shops from various online merchants and the Web interfaces of different \gls{PSP}s and issuers are an enormous undertaking and will likely not succeed due to different notions of the communication patterns and data structures exchanged between all relevant participants. \\

This shows the scenario this Master thesis will look into in detail and try to solve the most important question: \textit{is this really a fraudulent transaction?} \\

Looking into the stakeholders, that can provide useful information to decide on it, one will come up with:\@

\begin{enumerate}
    \item \textbf{merchant}, who can provide additional information of each E-commerce transaction in question
    \item \textbf{\gls{PSP}/issuer}, whose offer information about the credit card usage pattern and the original credit card owner
    \item \textbf{\gls{LSP}}, who can offer information about whether the order has already been shipped or not, and in the former case to whom it has been handed over
    \item \textbf{\gls{ISP}}, who can on request give hints whether a consumer has fallen victim to a phishing attack based on her Internet access logs
\end{enumerate}

% section scope_thesis (end)
