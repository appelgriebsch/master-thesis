%!TEX root = ../MasterThesis.tex

\section{Scope of this Master Thesis}
\label{sec:scope_thesis}

As laid out in the previous section, the most interesting \gls{E-commerce} fraud scenario is the one, in which fraudsters use leaked credit card information to order products or services from various merchants on the Internet. This is currently most likely to be successful, because there is a lack of information on the side of the merchants as well as the \gls{PSP}s and the issuers. Each of the affected merchants just noticed the transaction that takes place in their own Web shop, without knowing about the other attempts the fraudsters do on the Internet. The \gls{PSP}s and the issuers will both notice the active use of a credit card on different Web shops though, but do not have any transaction details. Therefore they could not correlate the data from these transactions to check for suspicious activities. \\

Based on the current credit card usage patterns of the fraudsters, who will try a leaked credit card in commonly used Web shops, it is more likely that these fraudulent transactions will not be recognized on time by the existing fraud prevention techniques in place. \\

A simple approach to solve these issues would be to just share more information of the ongoing transactions between the merchants, the \gls{PSP}s and the issuers. This approach might be subject to fail though, because adapting and harmonizing the communication interfaces between the Web shops from various online merchants and the Web Service interfaces of different \gls{PSP}s and issuers are an enormous undertaking. Any attempt will likely not succeed due to different notions of the communication patterns and data structures exchanged between all relevant participants. \\

To solve these problems this Master thesis will look into the information sharing issues in detail and try to come up with a solution to answer the most important question of this scenario: \@

\begin{quotation}
  \textit{Is this transaction really a valid \gls{E-commerce} transaction?}
\end{quotation}

Looking into the stakeholders, who can provide useful information to decide it, one will come up with:\@

\begin{enumerate}
    \item \textbf{Merchants}, who can provide additional information of each \gls{E-commerce} transaction in question.
    \item \textbf{\gls{PSP}s/issuers}, that have information about the credit card usage patterns and the original credit card owners.
    \item \textbf{\gls{LSP}s}, who can offer information about whether an order has already been shipped or not, and in the former case to whom it has been handed over.
\end{enumerate}

Its important to point out, that parts of the shared information are confidential or business-critical to at least one of the stakeholders involved. Due to this fact the data sharing has to be secured, and access to the resources has to be granted to selected participants of the scenario only. This Master thesis will focus on the data sharing, collecting and combining aspects of the collaborative system. A detailed discussion of the security aspects of it, incl.\ how to restrict access to the data with available techniques such as \gls{OAuth}, is out of scope of the thesis though. Additionally, as a \gls{CSCW} system \emph{always} consists of a social and a technological component introducing such a collaborative system in an existing organisation will raise issues of user acceptance and adaptation to business processes, that are also left out of the discussions in this Master thesis. 

% section scope_thesis (end)
