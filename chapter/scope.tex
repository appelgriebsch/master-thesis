%!TEX root = ../MasterThesis.tex

\section{Scope of this Master Thesis}
\label{sec:scope_thesis}

Workshop ErsteBank Wien: \\
- still this kind of fraud prevention mgmt. can not solve all issues due to the amount and frequency of the transactions, there is generally a
fraud-to-sales ratio of max. 0.11 percent in the EU (meaning 1 promille of transactions are fraudulent) \\

- ca. 85 percent of frauds are E-commerce frauds (EU: 70-90 percent). Hotspots are Germany, France and US. Frauds are coming from Travel-Shops or Online Merchants and the amount is on average between 500-600 EUR \\
- various regulations make it hard to shre detailed information with involved parties (even if they have special agreements signed between them) \\
- main questions for E-commerce fraud: who is the party that is the victim of the incident? Is it really a fraudulent transaction? \\
- E-commerce fraud can not be handled by technology alone, at best the fraudulent transaction can be blocked on the merchant side (due to the information given by the consumer like items, prices, delivery address, ...) \\
- in the worst case one successful fraudulent transaction in an E-commerce shop will trigger hundred and thousands of attempts -> so the awareness for the issue has to be at merchant side \\
- at the end: much effort is assumed to bring all the experts together and solve the issue by putting their individual know-how on the table \\



% section scope_thesis (end)
