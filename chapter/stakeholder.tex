%!TEX root = ../MasterThesis.tex

\section{Stakeholders}
\label{sec:stakeholder_analysis}

The following section looks at each stakeholder involved in the \gls{E-commerce} scenario in detail, lists the kind of information they own or provide to others, as well as describes the role of each stakeholder in the \gls{E-commerce} fraud investigation process (if any).

\subsection{Consumer}
\label{subsec:stakeholder_consumer}

The consumers are the initiators of \gls{E-commerce} transactions. They use the shop of a merchant on the Internet to order products or services. For doing so they have to know the \gls{URL} of the Web shop, have to be connected to the Internet via an \gls{ISP}, and have to use a standard software called a Web browser on their computer. For the duration of their online browsing sessions they also own a unique \gls{IP} address provided by the \gls{ISP}.\\

They might have had a long-term business relationship with the merchant and already own an user account on the Web shop. As an alternative they might be just interested into a one-time shopping trip, and might want to order the items without creating an account first --- sometimes also called ``anonymous'' or ``guest'' checkout in the \gls{E-commerce} shops. \\

The consumers also have a bank account, and at least own a debit card from that bank to get access to the money on the account. In addition to that they can also hold multiple credit cards. A credit card can be issued by the same bank, or can be provided by another financial service institution (e.g.\ American Express). In any case the organization that has handed out the credit card to the consumer is called the issuer. \\

If the consumers are going to order items in a Web shop, they will usually browse the product and service offerings of a merchant and put the items of interest into the shopping cart first. During the completion of the transaction they have to state the following information to the merchant:\@

\begin{itemize}
		\item personal information \gls{incl}\ given name, family name and date of birth,
		\item the address the items should be shipped to,
		\item payment information \gls{incl}\ type of payment and billing address (if different to shipping address).
\end{itemize}

If they are going to end the transaction with a payment of type credit card, they will also have to provide specific information of the credit card that should be used as payment:\@

\begin{itemize}
		\item the owner of the credit card (if it is not belonging to themselves),
		\item the unique credit card number,
		\item the expiry date of the credit card (in format MM/YY),
		\item the security code of the credit card.
\end{itemize}

The consumers have a special role in the whole scenario. As the online merchants have to deal with the consumers without any face-to-face or real-world interactions, the consumers are also the least trustworthy participants from the point of view of the merchants. As Section~\ref{sec:scenario_fraud} will show, the consumers are the main subjects of investigation in the case of an \gls{E-commerce} fraud incident. Therefore, the consumers are not taking any active part in the fraud investigation process.

% subsection stakeholder consumer

\subsection{Merchant}
\label{subsec:stakeholder_merchant}

The merchants offer products and services on the Internet to the general public. They might use the Internet as an additional sales channel, or rely on it solely for making any business. To provide access to the Web shop a merchant has to register a domain name and a \gls{URL} with a local domain name registry. This specific \gls{URL} refers to a fixed public \gls{IP} address, which the server that runs the Web shop software uses. Normally the merchants do not operate the servers themselves, but rely on a service offered by a Cloud Service Provider for that purpose. Moreover, the Web shop software itself is usually not provided by the merchants, but bought from an \gls{ISV} on the market. In any case the merchants have special responsibilities in the Web shop, because they have to take care to configure the products, prices, promotions, payment, and shipment services available. In addition products can be categorized by them into categories and sub-categories for easier navigating and searching the offerings in the Web shop by the consumers later. \\

The merchants can decide whether they restrict ordering of products to registered users only, or allow anonymous users too. The main benefit of the former is the possibility to analyse the shopping behaviour of individual consumers, whereas the latter will open the business for a wider range of consumers as it includes also those, who do not want to register with any existing online shop. Nevertheless, any consumer activity on the online shop is tracked in the analytic databases of a merchant. This includes not only the items that have been placed into the shopping cart, but also any product that a consumer has looked at during a shopping session. Even though these detailed analytic capabilities are actually synonymous for their usage in target-related advertising, they can also help to decide whether a consumer behaves normally or not within a Web shop. \\

Any business transaction that a consumer makes with a merchant is stored in the merchant’s database. The transaction information contains, but is not limited to:\@

\begin{itemize}
		\item personal related information of the consumer,
		\item the address the items will be shipped to,
		\item a collection of products with quantities and prices,
		\item the total amount of the order considering promotions, taxes and fees,
		\item the selected payment information.
\end{itemize}

If a consumer wants to pay with credit card, the payment process is not handled by the merchants themselves, but is routed to a Payment Service Provider (\gls{PSP}) on the Internet instead. To initiate the credit card authorization a merchant is sending a request with the following information to the Web service endpoint of a \gls{PSP}: \@

\begin{itemize}
    \item consumer's billing address,
    \item given credit card number, expiry date and security code,
    \item identification of the merchant,
    \item final amount of the current transaction.
\end{itemize}

In return of the payment authorization a merchant receives and stores these  payment-related information for the transaction:\@

\begin{itemize}
		\item the type of credit card used (e.g.\ Visa, MasterCard, American Express, \ldots),
		\item the name of the credit card owner,
		\item the unique payment token received by the \gls{PSP},
		\item the timestamps and result code of the authorization,
		\item the authority, who has approved the payment (if the merchant works with multiple Payment Service Providers).
\end{itemize}

As the merchants will collect a lot of personal and payment-related information over time, they are also one of the major sources of possible data leaks in the \gls{E-commerce} scenario. Due to this circumstance the Payment Card Initiative, a group of banks, issuers and \gls{PSP}s, provides rules and guidelines (aka \gls{PCI/DSS} standards) for securely handling these kind of information in an \gls{IT} system \citep{virtue2009payment}. \\

The merchants are one of the main actors in the fraud investigation process. They are highly interested in figuring out whether a consumer's transaction is valid or not. This is due to the fact that in case of an \gls{E-commerce} fraud incident the merchants will likely have to cover the costs (see Section~\ref{sec:scenario_fraud}). Also the online merchant's reputation will suffer, if private information from their databases gets leaked. If a merchant falls victim to fraud incidents multiple times, the economic damages can finally result in a bankruptcy of that merchant.

% subsection stakeholder merchant

\subsection{Payment Service Provider}
\label{subsec:stakeholder_psp}

The Payment Service Providers offer payment-related services to online merchants. To be able to do this a \gls{PSP} provides a Web service interface, which the merchants have to communicate with by sending payment authorization requests to it (see above). The \gls{PSP}s might be able to authorize the payment requests on their own, or might have to route each of them to the corresponding issuer of the credit card used. For the former procedure the \gls{PSP}s have to run a database of registered users with credit card information on their own (e.g.\ a Web service such as PayPal). For the latter they will have to know the issuer of the credit card used, and have to call into the Web service of that issuer for validation purposes. For verifying a credit card and authorizing the payment a merchant hands over the following:\@

\begin{itemize}
		\item credit card owner \gls{incl}\ billing address given,
		\item credit card number,
		\item credit card expiry date,
		\item credit card security code,
		\item identification of the merchant,
		\item total amount of the current transaction.
\end{itemize}

In case the \gls{PSP}s authorize the payment requests themselves, they will have to securely process the information and return the validation results to the merchants. Each result message also contains a unique payment token that a merchant can refer to later to initiate the clearing process. As of this, the \gls{PSP}s have to persist the credit card and payment-related information in their databases. According to industry standards they should also follow the \gls{PCI/DSS} guidelines mentioned in the previous section. \\

The level of activity in the \gls{E-commerce} fraud investigation process depends on whether the \gls{PSP}s authorize the payments themselves, or only act as a routing service between the merchants and the original credit card issuers. In the former case the \gls{PSP}s are more actively involved. In that situation they also hold more of the valuable information to analyse an \gls{E-commerce} fraud incident. In the latter case they will still be required to connect the payment-related request information from a merchant with the corresponding authorization result coming from an issuer. \\

If the \gls{PSP}s hold sensitive information in their databases, they will also be a source of possible data leaks. In that situation they have to put the same precautions in place as issuers have to do (as explained in the next section).

% subsection stakeholder psp

\subsection{Issuer}
\label{subsec:stakeholder_issuer}

The issuers are the only members in the \gls{E-commerce} scenario that know the owners of credit cards in person. Each individual has to register personally with an issuer to get access to a credit card. This registration process includes providing the following information: \@

\begin{itemize}
		\item personal related information such as given name, family name and date of birth,
		\item the currently registered home address,
		\item the bank account that should be used to settle credit card balances.
\end{itemize}

Even if the two parties do not really meet each other personally, individuals will still have to identify themselves with a valid ID card and bank account to receive and activate a new credit card. Beside being the single source of truth about the original credit card owner, the issuers of credit cards also collect and store all of their usages. Whereas the Payment Service Providers can only provide individual credit card usage patterns for the online shopping scenario, the issuers can also include those transactions that the credit card owners do in the real-world. Needless to say that these are valuable information for an \gls{E-commerce} fraud investigation. \\

Still an issuer does not know any details of the transactions that have been made with a credit card yet. As shown in the Section~\ref{subsec:stakeholder_psp} the issuers receive only an identifier of the merchants a credit card has been used with. Based on public available information about merchants in a commercial register the issuers could come up with at least the retail branch each merchant operates in. \\

Being the single source of truth about all issued credit cards, their owners and usage patterns make the issuers another high-risk candidates for possible data leaks. They should as well follow the guidelines from the \gls{PCI/DSS} standards, should incorporate security standards for their \gls{IT} systems and the processes of operating them, as well as monitor their \gls{IT} systems actively with an intrusion detection mechanism.

% subsection stakeholder issuer

\subsection{Acquirer}
\label{subsec:stakeholder_acquirer}

The acquirers hold the bank accounts of merchants and are responsible for withdrawing the outstanding amounts of transactions from the accounts of the consumers, or more precisely requesting it from the issuer of each consumer. Due to this an acquirer usually does not process any credit card related information from consumers directly, but refers to the unique payment tokens that have been created by the \gls{PSP}s or the issuers during the authorization processes. \\

Still as financial institutions acquirers (like issuers) have to comply with the rules and guidelines of the \gls{PCI/DSS} and other industry standards to make sure that their bank accounts as well as the transaction processing are safe and secure. The detailed analysis of these techniques and procedures as well as possible banking fraud incidents are out of scope of this Master Thesis though.

% subsection stakeholder acquirer

\subsection{Logistic Service Provider}
\label{subsec:stakeholder_lsp}

The Logistic Service Providers have two important roles in the \gls{E-commerce} scenario. First, they have access to and control over the items of a merchant for the duration of the transport between the merchant's facility and the consumer's shipping address. And second, they hold the information to whom they have handed over the items at the final destination. Although the \gls{LSP}s have nothing to do with any payment-related activities, they are still critical actors for the investigation of fraud incidents, because they will be the last chance for a merchant to stop the delivery of an order (in case a fraud has been detected after initiating the shipment), or provide information about the person  that has received the items at the shipping address --- especially so for orders of high-priced goods, which usually require a recipient to identify with a personal ID card and place a signature on the delivery receipt. \\

For initiating the shipment procedure a merchant orders a certain transport service from a \gls{LSP}, and hands over the following information:\@

\begin{itemize}
	\item name of the recipient,
	\item delivery address given by the consumer,
	\item list of items to be shipped,
	\item optionally: value of the items if an insurance policy is taken.
\end{itemize}

The \gls{LSP} returns a unique tracking number for the shipment in response. This number can be used by the merchant, and the consumer, to check for the status of a shipment online. \\

As the \gls{LSP}s do not have to deal with the payment-related activities in the \gls{E-commerce} scenario, they are also not actively involved in the fraud investigation. However, they can stop the delivery of the items, or provide useful information about the recipients if an incident is found.

% subsection stakeholder lsp

\subsection{Cloud Service Provider}
\label{subsec:stakeholder_csp}

The Cloud Service Providers offer \gls{IT} services to their customers. These \gls{IT} services include hardware and software assets that merchants can order in the \gls{E-commerce} scenario to run their Web shops on the Internet. Part of the service level agreement between a merchant and a \gls{CSP} is a detailed listing of the responsibilities of both parties (who has to take care of what). In most cases the merchants are outsourcing the complete operation of the hardware and software for their Web shops to the \gls{CSP}s; so the \gls{CSP}s are responsible for making sure that the Web shops are available and secure. The \gls{CSP}s are also constantly monitoring the incoming connections to each public Internet server under their control and can provide information whether a Web shop of one of the merchants has been compromised or not. However, the \gls{CSP}s are not actively involved in the \gls{E-commerce} fraud investigation.

% subsection stakeholder csp

\subsection{Independent Software Vendor}
\label{subsec:stakeholder_isv}

The Independent Software Vendors design, implement and sell the Web shop software tools. They have detailed knowledge about the software components and libraries used within their Web shop products and check them regularly for security breaches or vulnerabilities. They also have to verify these software parts for vulnerabilities that they have implemented on their own, as well as have to make sure that their implementations follow industry standards (e.g.\ \gls{PCI/DSS} for handling person and payment-related information). Therefore, they can best assert these quality criteria of a Web shop software if needed. Thus, the \gls{ISV}s are not an active member of an \gls{E-commerce} fraud investigation.

% subsection stakeholder isv

\subsection{Internet Service Provider}
\label{subsec:stakeholder_isp}

The Internet Service Providers offer services to the consumers so that they are able to connect to and make use of the Internet. Each Web request consumers are doing on their systems is routed to the Internet via the infrastructure of an \gls{ISP}. Due to existing regulations and laws the \gls{ISP}s have to store the log files of each Internet session of their customers for a certain amount of time. Especially, these log files can be helpful to decide whether a consumer was visiting pages in the dark-side of the Web, or if they fall victim to some phishing attacks (explained later in Section~\ref{sec:scenario_fraud}). Although these information can be helpful to decide on fraudulent transactions in the \gls{E-commerce} scenario, the \gls{ISP}s are not actively involved in the investigation of it. They are rather required for getting information about the deceivers in case a fraud has been detected.

% subsection stakeholder isp

% section stakeholder analysis
