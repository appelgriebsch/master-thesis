%!TEX root = ../MasterThesis.tex

\section{Master Thesis Outline}
\label{sec:thesis_outline}

Before starting with the investigation of \gls{E-commerce} fraud incidents and their possible examinations, the thesis starts with an analysis of related works in Chapter~\ref{cha:related_works}, that have been looked into during the course of this Master thesis and have had an influence on it. \\

In the next part, Context Analysis in Chapter~\ref{cha:context_analysis}, the thesis discusses the \gls{E-commerce} scenario in detail. It starts with a description of the \gls{E-commerce} shopping process, looks into the stakeholders involved as well as shows possible kinds of \gls{E-commerce} fraud incidents and how they are handled today. Based on these discussions this chapter closes with a presentation of the specific scenario, that has been selected for further examination within this Master thesis. \\

After this initial scope setup the thesis briefly outlines the theoretical foundations required for the understanding of the concepts in the solution space in Chapter~\ref{cha:theoretical_foundations}. This section starts with a short overview of the relevant facets of computer-supported collaborative work systems (\gls{CSCW}), shows the essential specifications of the Semantic Web, and ends up with an introduction to the peer-to-peer (\gls{P2P}) communication techniques and protocols. \\

In the main part of this thesis (Chapter~\ref{cha:design_system}) a collaborative system, that supports the investigation of \gls{E-commerce} fraud incidents, is developed. This chapter lays out and discusses the possibilities for designing and using such a system. The objective is to come up with an approach at the end of this chapter, that might be the best fit for the problem described in the scenario at the beginning. \\

To conclude the thesis also sum up the findings and give an outlook of future work on this topic.

% section thesis outline (end)
