%!TEX root = ../MasterThesis.tex

\section{Outline of this Master Thesis}
\label{sec:thesis_outline}

Before looking into the investigation of \gls{E-commerce} fraud incidents and their possible examinations, the Master Thesis starts with a description of related works in Chapter~\ref{cha:related_works}. These research papers have been evaluated during the course of this Master Thesis, and have had an influence on the concept and design of the collaborative system being developed. \\

In the next part, Context Analysis in Chapter~\ref{cha:context_analysis}, the Master Thesis discusses the \gls{E-commerce} scenario in detail. It starts with a description of the \gls{E-commerce} shopping process, looks into the stakeholders involved, as well as shows possible kinds of \gls{E-commerce} fraud incidents and how they are handled today. Based on these findings this chapter closes with a presentation of the specific scenario that has been selected for further examination within this Master Thesis. \\

After this initial scope setup the Master Thesis briefly outlines the theoretical foundations in Chapter~\ref{cha:theoretical_foundations}, which are required for the understanding of the concepts and design decisions in Chapter~\ref{cha:system_concept} and Chapter~\ref{cha:system_design}. This section starts with a short overview of the relevant facets of computer-supported collaborative work systems (\gls{CSCW}), shows the essential specifications of the Semantic Web, and ends up with an introduction to the peer-to-peer (\gls{P2P}) communication techniques and protocols. \\

The main parts of this Master Thesis (Chapter~\ref{cha:system_concept} and Chapter~\ref{cha:system_design}) discuss the concept and design for a collaborative system that supports the investigation of \gls{E-commerce} fraud incidents. These chapters will elaborate and analyse the possibilities for designing and using such a collaborative system. The objective is to come up with an approach at the end of the discussions that might be the best fit for the problem described in the scenario in Chapter~\ref{cha:context_analysis}. \\

To conclude the Master Thesis also sums up the findings and gives an outlook for future work on this topic.

% section thesis outline (end)
