%!TEX root = ../MasterThesis.tex

\section{Master Thesis Outline}
\label{sec:thesis_outline}

In the first section this Master thesis will discuss the E-commerce scenario in detail. It starts with a description of the E-commerce shopping process, looks into the stakeholders involved and further shows possible attacks and vulnerabilities. Based on this discussion the chapter will close with a presentation of the specific scenario selected for further investigation. \\

After this initial scope setup the thesis will briefly outline the theoretical foundations required for the understanding of the concepts in the solution space. This section starts with a short overview of the importants aspects of computer-supported collaborative work systems, shows the needed technical specifications of the Semantic Web standards and ends up with an introduction to the peer-to-peer communication techniques and protocols. \\

Before starting with the investigation of possible solutions for the problem described in the end of the first section the thesis will also list related works, that has been examined in the course of this Master thesis and have had an influence on the solution space. \\

Last but not least the conceptualization of a collaborative system that supports the investigation of E-commerce fraud takes place. This chapter will lay out and discuss the possibilities for designing and using such a system. The objective is to come up with an approach at the end of this chapter, that might be the best fit for the problem described in the first section. \\

To conclude the thesis will also show an outcome of the paper work as well as give an outlook that might be useful to decide future progress on this topic.

% section thesis outline (end)
