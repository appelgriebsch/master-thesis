%!TEX root = ../MasterThesis.tex

\section{Master Thesis Outline}
\label{sec:thesis_outline}

Before starting with the investigation of E-commerce fraud incidents and their possible examinations the thesis will first analyse related works, that have been looked into during the course of this Master thesis and have had an influence on it. \\

In the next section, Context Analysis, the thesis will discuss the E-commerce scenario in detail. It starts with a description of the E-commerce shopping process, looks into the stakeholders involved, shows possible kinds of E-commerce fraud incidents and how they are handled today. Based on these discussions the chapter will close with a presentation of the specific scenario selected for further examination within this Master thesis. \\

After this initial scope setup the thesis will briefly outline the theoretical foundations required for the understanding of the concepts in the solution space. This section starts with a short overview of the important facets of computer-supported collaborative work systems (\gls{CSCW}), shows the essential  specifications of the Semantic Web and ends up with an introduction to the peer-to-peer (\gls{P2P}) communication techniques and protocols. \\

Last but not least the conceptualization of a collaborative system, that supports the investigation of E-commerce fraud incidents, takes place. This chapter will lay out and discuss the possibilities for designing and using such a system. The objective is to come up with an approach at the end of this chapter, that might be the best fit for the problem described in the scenario at the beginning. \\

To conclude the thesis will also show an outcome of the paper work as well as give an outlook, that might be beneficial to decide future progress on this topic.

% section thesis outline (end)
