\documentclass[BCOR=12mm,DIV11,titlepage,a4paper,oneside]{scrbook}

% package for encoding, here UTF-8
\usepackage[utf8]{inputenc}

% package for including graphics with figure-environment
\usepackage{graphicx}

% package for changing header and footer
\usepackage{fancyhdr}
% Uses the package for own page style
\pagestyle{fancy}
% Creates a line in header (to hide this, change to 0.0pt)
\renewcommand*{\headrulewidth}{0.4pt}
\fancyhf{}
\fancyhead[EC,OC]{\thepage}
% \fancyhead[EL]{\leftmark}
% \fancyhead[OR]{\rightmark}
% \fancyhead[ER,OL]{\thepage}
\renewcommand{\sectionmark}[1]{
\markboth{\thechapter{} #1}{\thechapter{} #1}
}

% changes page numbering of table of content with own style
\renewcommand*{\indexpagestyle}{fancy}
% prevents page numbering on "Part" pages
\renewcommand*{\partpagestyle}{empty}
% changes page numbering of chapters with own style
\renewcommand*{\chapterpagestyle}{fancy}

% Changes numbering of figures (chapter-dependent, e.g. figure 1.1)
\renewcommand*{\thefigure}{\thechapter.\arabic{figure}}
% Changes numbering of tables (table-depentent, e.g. table 1.1)
\renewcommand*{\thetable}{\thechapter.\arabic{table}}

% prevents indent after section and figures
\setlength{\parindent}{0pt}

% prevents that a new page is created for a single word / row
\clubpenalty = 10000 % exclusion of orphans
\widowpenalty = 10000 % exclusion of widow lines

% package for bibliography
\usepackage[authoryear,round]{natbib}
\bibliographystyle{natdin}

% package for the use of URL by the use of the command \url{}
\usepackage{url}

% package for line spacing (onehalfspace, singlespace)
\usepackage{setspace}

% package for the creation of quotation marks by the use of the command \enquote{Text}
\usepackage[english]{babel}
\usepackage[babel, german=quotes]{csquotes}

% package for colored text
% black,white,green,red,blue,yellow,cyan,magenta
\usepackage{color}

% package for colored tables
\usepackage{colortbl}

% Rotation of floating objects (figures, tables, etc.)
\usepackage{rotating}

% Rotation of single pages by the use of the command begin{landscape}
\usepackage{lscape}

% package for colored background Verbatim-environment (source code environment)
\usepackage{fancyvrb}
\usepackage{verbatim,moreverb}
% defindes shade of grey for source code environment 80 % grey
\definecolor{sourcegray}{gray}{.80}

% package for source code environment
\usepackage{listings}

% package for positioning objects without float (Example of usage: \begin{figure}[H])
\usepackage{here}

% Alternatives Paket für here.sty
% \usepackage{float}

% Package for the creation of hyperrefs and pdf informations
\usepackage[pdftex,plainpages=false,pdfpagelabels,
            pdftitle={title},
            pdfauthor={name}
            ]{hyperref}

% colors for hyperlinks
% colored borders (false) colored text (true)
\hypersetup{colorlinks=true,citecolor=black,filecolor=black,linkcolor=black,urlcolor=black}

% two directories for "content" and "appendix"
\usepackage{appendix}
