\documentclass[BCOR=12mm,DIV11,titlepage,a4paper,oneside]{scrbook}

% package for encoding, here UTF-8
\usepackage[utf8]{inputenc}

% package for including graphics with figure-environment
\usepackage{graphicx}

% package for changing header and footer
\usepackage{fancyhdr}
% Uses the package for own page style
\pagestyle{fancy}
% Creates a line in header (to hide this, change to 0.0pt)
\renewcommand*{\headrulewidth}{0.4pt}
\fancyhf{}
\fancyhead[EC,OC]{\thepage}
% \fancyhead[EL]{\leftmark}
% \fancyhead[OR]{\rightmark}
% \fancyhead[ER,OL]{\thepage}
\renewcommand{\sectionmark}[1]{
\markboth{\thechapter{} #1}{\thechapter{} #1}
}

% changes page numbering of table of content with own style
\renewcommand*{\indexpagestyle}{fancy}
% prevents page numbering on "Part" pages
\renewcommand*{\partpagestyle}{empty}
% changes page numbering of chapters with own style
\renewcommand*{\chapterpagestyle}{fancy}

% Changes numbering of figures (chapter-dependent, e.g. figure 1.1)
\renewcommand*{\thefigure}{\thechapter.\arabic{figure}}
% Changes numbering of tables (table-depentent, e.g. table 1.1)
\renewcommand*{\thetable}{\thechapter.\arabic{table}}

% prevents indent after section and figures
\setlength{\parindent}{0pt}

% prevents that a new page is created for a single word / row
\clubpenalty = 10000 % exclusion of orphans
\widowpenalty = 10000 % exclusion of widow lines

% package for bibliography
\usepackage[authoryear,round]{natbib}
\bibliographystyle{natdin}

% package for the use of URL by the use of the command \url{}
\usepackage{url}

% package for line spacing (onehalfspace, singlespace)
\usepackage{setspace}

% package for the creation of quotation marks by the use of the command \enquote{Text}
\usepackage[english]{babel}
\usepackage[babel, german=quotes]{csquotes}

% package for colored text
% black,white,green,red,blue,yellow,cyan,magenta
\usepackage{color}

% package for colored tables
\usepackage{colortbl}

% Rotation of floating objects (figures, tables, etc.)
\usepackage{rotating}

% Rotation of single pages by the use of the command begin{landscape}
\usepackage{lscape}

% package for colored background Verbatim-environment (source code environment)
\usepackage{fancyvrb}
\usepackage{verbatim,moreverb}
% defindes shade of grey for source code environment 80 % grey
\definecolor{sourcegray}{gray}{.80}

% package for source code environment
\usepackage{listings}

% package for positioning objects without float (Example of usage: \begin{figure}[H])
\usepackage{here}

% Alternatives Paket für here.sty
% \usepackage{float}

% Package for the creation of hyperrefs and pdf informations
\usepackage[pdftex,plainpages=false,pdfpagelabels,
            pdftitle={title},
            pdfauthor={name}
            ]{hyperref}

% colors for hyperlinks
% colored borders (false) colored text (true)
\hypersetup{colorlinks=true,citecolor=black,filecolor=black,linkcolor=black,urlcolor=black}

% two directories for "content" and "appendix"
\usepackage{appendix}

\usepackage[nonumberlist]{glossaries}
\makeglossaries
%!TEX root = ../MasterThesis.tex

\newglossaryentry{CSCW}{
  name={CSCW},
  description={computer-supported cooperative work}
}

\newglossaryentry{XMPP}{
  name={XMPP},
  description={Extensible Messaging and Presence Protocol}
}

\newglossaryentry{WebRTC}{
  name={WebRTC},
  description={Web Real-Time Communication}
}

\newglossaryentry{RDF}{
  name={RDF},
  description={Resource Description Framework}
}

\newglossaryentry{RDFa}{
  name={RDFa},
  description={Resource Description Framework in Attributes}
}

\newglossaryentry{RDFS}{
  name={RDFS},
  description={Resource Description Framework Schema}
}

\newglossaryentry{OWL}{
  name={OWL},
  description={Web Ontology Language}
}

\newglossaryentry{SPARQL}{
  name={SPARQL},
  description={SPARQL Protocol and RDF Query Language}
}

\newglossaryentry{W3C}{
  name={W3C},
  description={World-Wide Web Consortium}
}

\newglossaryentry{XML}{
  name={XML},
  description={Extensible Markup Language}
}

\newglossaryentry{URL}{
  name={URL},
  description={Uniform Resource Locator}
}

\newglossaryentry{P2P}{
  name={P2P},
  description={Peer-To-Peer}
}

\newglossaryentry{PSP}{
  name={PSP},
  description={Payment Service Provider}
}

\newglossaryentry{ISV}{
  name={ISV},
  description={Independent Software Vendor}
}

\newglossaryentry{ISP}{
  name={ISP},
  description={Internet Service Provider}
}

\newglossaryentry{CSP}{
  name={CSP},
  description={Cloud Service Provider / Hosting Service}
}

\newglossaryentry{LSP}{
  name={LSP},
  description={Logistic Service Provider}
}

\newglossaryentry{B2B}{
  name={B2B},
  description={Business-To-Business}
}

\newglossaryentry{B2C}{
  name={B2C},
  description={Business-To-Consumer}
}

\newglossaryentry{C2B}{
  name={C2B},
  description={Consumer-To-Business}
}

\newglossaryentry{C2C}{
  name={C2C},
  description={Consumer-To-Consumer}
}

\newglossaryentry{IP}{
  name={IP},
  description={Internet Protocol}
}

\newglossaryentry{API}{
  name={API},
  description={Application Programming Interface}
}

\newglossaryentry{HTML}{
  name={HTML},
  description={HyperText Markup Language}
}

\newglossaryentry{SGML}{
  name={SGML},
  description={Standard Generalized Markup Language}
}

\newglossaryentry{HTTP}{
  name={HTTP},
  description={HyperText Transfer Protocol}
}

\newglossaryentry{REST}{
  name={REST},
  description={Representational state transfer}
}

\newglossaryentry{URI}{
  name={URI},
  description={Uniform Resource Identifier}
}

\newglossaryentry{JSON-LD}{
  name={JSON-LD},
  description={JavaScript Object Notation for Linked Data}
}

\newglossaryentry{ETL}{
  name={ETL},
  description={Extract-Transform-Load}
}

